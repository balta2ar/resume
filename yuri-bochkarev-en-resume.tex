% LaTeX source of my resume
% =========================

% Commented for easy reuse... ;)

% See the `README.md` file for more info.

% This file is licensed under the CC-NC-ND Creative Commons license.


% start a document with the here given default font size and paper size
%\documentclass[10pt,a4paper]{article}
%\usepackage[T2A]{fontenc}
%\usepackage[russian,english]{babel}
%\usepackage[utf8x]{inputenc}

%\usepackage[cm-default]{fontspec}
%\usepackage{xltxtra}

%\usepackage[xetex]{graphicx}
%\usepackage{fontspec,xunicode}

%\defaultfontfeatures{Mapping=tex-text,Scale=MatchLowercase}

\documentclass[unicode, 10pt, a4paper, oneside, fleqn]{article}

%\usepackage[T1]{fontenc}

\usepackage{polyglossia}  %% подключает пакет многоязыкой вёрстки
\setmainfont[Ligatures=TeX]{Liberation Serif}  %% задаёт основной шрифт документа
%\setmainfont[Ligatures=TeX]{Linux Libertine O}  %% задаёт основной шрифт документа
%\setmainfont{PT Serif}  %% задаёт основной шрифт документа
%\setsansfont{Liberation Sans}  %% задаёт шрифт без засечек
%\setmonofont{Liberation Sans Mono}  %% задаёт моноширинный шрифт

%\setmainfont[Mapping=tex-text, Numbers=OldStyle, Ligatures={Common,Contextual}]{Linux Libertine O}

\defaultfontfeatures{Scale=MatchLowercase, Mapping=tex-text}  %% устанавливает поведение шрифтов по умолчанию
\setdefaultlanguage[spelling=modern]{russian}  %% устанавливает язык по умолчанию
\setotherlanguage{english}

%\fontsize{4mm}{5mm}\selectfont

%\renewcommand{\normalsize}{\fontsize{12.5}{10}\selectfont}

%\defaultfontfeatures{Mapping=tex-text,Scale=MatchLowercase}
%\setmainfont[Scale=.95]{Arial}
%\setmonofont{DejaVu Sans}

\usepackage{comment}

% include the `tex` instructions that takes care of loading packages and defining commands
\include{resume-commands}

\begin{document}  % begin the content of the document
\sloppy  % this to relax whitespacing in favour of straight margins

\maintitle{Yuri Bochkarev}{May 24, 1988}  % title on top of the document

\nobreakvspace{0.3em}  % add some page break averse vertical spacing

% \noindent prevents paragraph's first lines from indenting
% \mbox is used to obfuscate the email address
% \sbull is a spaced bullet
% \href well..
% \\ breaks the line into a new paragraph
\noindent\href{mailto:baltazar.dot.bz.at.gmail.dot.com}{baltazar.bz\mbox{}@\mbox{}gmail.com}\sbull
\textsmaller{+}79853037426\sbull
%cies010 \emph{(Skype)}\sbull
\href{http://ru.linkedin.com/pub/yuri-bochkarev/21/3a9/555}{linkedin}
\sbull
\href{https://bitbucket.org/balta2ar}{bitbucket}
\sbull
\href{https://github.org/balta2ar}{github}
\sbull
\href{http://baltazar-bz.blogspot.com/}{blog}
\\
%Mathenesserplein 84\sbull
%3022\thinspace {\sc ld}\sbull
Moscow\sbull
Russia
\spacedhrule{0.9em}{-0.4em}  % a horizontal line with some vertical spacing before and after

\roottitle{Summary}  % a root section title

\vspace{-1.3em}  % some vertical spacing

\begin{multicols}{2}  % open a multicolumn environment
\noindent 
1994 -- 2003: finished school (external studies).\\
2003 -- 2009: got bachelor's \& master's degrees in Ulyanovsk Technical State
University. \\
I work in Moscow since 2009.
\end{multicols}
\spacedhrule{0em}{-0.4em}

\roottitle{Experience}

\headedsection  % sets the header for the section and includes any subsections
  {\href{http://www.smartlabs.tv}{SmartLabs \acr{LLC}}}
  {\textsc{Moscow}} {%

  \headedsubsection  % sets the header for a subsection and contains usually body text
    {Developer in System Department}
    {September 2009 -- present}
    {\bodytext{
        \begin{itemize}
            \item{Development of user requests for video content viewing redirector (\CPP/Qt)}
            \item{Development of tool destined for video content distribution to client servers automation (\CPP/Qt)}
            \item{Integration of automation tool with existing platform}
            \item{Participation in Smart Media platform video server development
                  (\CPP/Qt, STL)}
              \item{Implementation of adaptive streaming (client side) (Apple HTTP Streaming)}
            \item{Development of tool for platform components assembly automation (python)}
            \item{Accompanying protocols: \acr{HTTP}, \acr{XML RPC},
                  \acr{SOAP}, \acr{RTSP}}
        \end{itemize}
    }}
}

\headedsection
{ \guillemotleft Autoscan\guillemotright {LLC}} 
  {\textsc{Ulyanovsk}} {
  \headedsubsection
    {Developer}
    {September 2007 -- May 2009}
    {\bodytext{
        \begin{itemize}
            \item{Development of network applications for real-time video transmission (\CPP/Qt)}
            \item{Automation of application's testing (AutoIt, erlang)}
            \item{Accompanying protocols: \acr{XMPP}, \acr{SIP}, \acr{RTP}, \acr{STUN}}
        \end{itemize}
    }}
}

\headedsection
  {\href{http://www.ulstu.ru}{Ulyanovsk Technical State University}}
  {\textsc{Ulyanovsk}} {
  \headedsubsection
    {Technician}
    {September 2006 -- September 2007}
    {\bodytext{
        \begin{itemize}
            \item{Administration of class (11 computers with servers based on \acr{ASP} Linux 9.0 и Microsoft Windows 2003})
            \item{Technical Support \& user consultation}
        \end{itemize}
    }}
}

\spacedhrule{0.5em}{-0.4em}
\roottitle{Education}

\headedsection
  {Ulyanovsk Technical State University}
  {\textsc{Ulyanovsk}} {
  \headedsubsection
    {Master degree}
    {2007 -- 2009} {
        \bodytext{Faculty of Information Systems \& Technologies, diploma \acr{VMA} 0105774}
    }
  }

\headedsection
  {Ulyanovsk Technical State University}
  {\textsc{Ulyanovsk}} {
  \headedsubsection
    {Bachelor degree}
    {2003 -- 2007} {
        \bodytext{Faculty of Information Systems \& Technologies, diploma \acr{AVB} 0512119}
    }
  }

\spacedhrule{0.5em}{-0.4em}
\roottitle{Additional Education}

\headedsection
  {\href{http://udacity.com}{udacity.com}}
  {} {
  \headedsubsection
    {\href{https://docs.google.com/document/d/1LpUyUwh_gGyPyKf-oxTDOy8ncQejwog1jhgMmtf59mY/edit}
          {Programming a Robotic Car}}
    {April, 2012} {}
  \headedsubsection
    {\href{https://docs.google.com/document/d/1wD_QEJ7mdzxbR_PMVEbZ_tZ0SyakJ_8Y1gBAj_S5Ufg/edit}
          {Algorithms: Design and Analysis, Part I}}
    {April, 2012} {}
  }
\headedsection
  {\href{http://www.coursera.org}{coursera.org}}
  {} {
  \headedsubsection
    {\href{https://docs.google.com/document/d/11OT8thqIgBiwM80D_HjpiGtKTz5CnxiITPG_H6QbuUA/edit}
          {Machine Learning}}
    {December, 2011} {}
  \headedsubsection
    {\href{https://docs.google.com/document/d/1wD_QEJ7mdzxbR_PMVEbZ_tZ0SyakJ_8Y1gBAj_S5Ufg/edit}
          {Articifial Intelligence}}
    {December, 2011} {}
  }

\spacedhrule{0.5em}{-0.4em}

\roottitle{Skills}

\inlineheadsection  % special section that has an inline header with a 'hanging' paragraph
  {Technical}
  {OS: Windows (95/98/ME/2000/XP/2003, 2001 -- 2009), Linux (ArchLinux, since 2009).
   Programming Languages: x86 assembler, C, C++, Python, Pascal, Object Pascal.
   Moderately experienced as well in: Haskell, Erlang, Clojure, Scala, Lua, AutoIt, \CS, Java,
   JavaScript. 
   Technologies: Test-Driven Development, Unit Testing, Design Patterns, UML.
  }
%  {Software design and implementation, with(in) a team. I love Ruby/\nsp
  %  Python/\nsp Java/\nsp \CPP and flirt regularly with Haskell. Solid
  %  knowledge of web technologies:\ \acr{HTML+CSS}, \acr{XML}, \acr{RDF},
  %  \acr{REST}, \acr{SOAP} and JavaScript (mainly jQuery). Linux administration
  %  skills:\ bash, Apache, My\acr{SQL}, Postgres\acr{SQL}, virtualization/cloud
  %  (Open\acr{VZ}, \acr{VM}ware, \acr{KVM}, Xen and \acr{EC}2), datacenter
  %  automation (Puppet and Chef), continuous integration (Hudson/Jenkins).}

\inlineheadsection
  {Language}
  {Russian \emph{(mother tongue)}, English \emph{(intermediate; fluent reading of technical documentation, technical writing).}}

\spacedhrule{1.6em}{-0.4em}

\roottitle{Interests}

\inlineheadsection
  {Bach fugues, bodybuilding, Haskell.}


\end{document}
