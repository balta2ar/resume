% LaTeX source of my resume
% =========================

% Commented for easy reuse... ;)

% See the `README.md` file for more info.

% This file is licensed under the CC-NC-ND Creative Commons license.


% start a document with the here given default font size and paper size
%\documentclass[10pt,a4paper]{article}
%\usepackage[T2A]{fontenc}
%\usepackage[russian,english]{babel}
%\usepackage[utf8x]{inputenc}

%\usepackage[cm-default]{fontspec}
%\usepackage{xltxtra}

%\usepackage[xetex]{graphicx}
%\usepackage{fontspec,xunicode}

%\defaultfontfeatures{Mapping=tex-text,Scale=MatchLowercase}

\documentclass[unicode, 10pt, a4paper, oneside, fleqn]{article}

%\usepackage[T1]{fontenc}

\usepackage{polyglossia}  %% подключает пакет многоязыкой вёрстки
\setmainfont[Ligatures=TeX]{Liberation Serif}  %% задаёт основной шрифт документа
%\setmainfont[Ligatures=TeX]{Linux Libertine O}  %% задаёт основной шрифт документа
%\setmainfont{PT Serif}  %% задаёт основной шрифт документа
%\setsansfont{Liberation Sans}  %% задаёт шрифт без засечек
%\setmonofont{Liberation Sans Mono}  %% задаёт моноширинный шрифт

%\setmainfont[Mapping=tex-text, Numbers=OldStyle, Ligatures={Common,Contextual}]{Linux Libertine O}

\defaultfontfeatures{Scale=MatchLowercase, Mapping=tex-text}  %% устанавливает поведение шрифтов по умолчанию
\setdefaultlanguage[spelling=modern]{russian}  %% устанавливает язык по умолчанию
\setotherlanguage{english}

%\fontsize{4mm}{5mm}\selectfont

%\renewcommand{\normalsize}{\fontsize{12.5}{10}\selectfont}

%\defaultfontfeatures{Mapping=tex-text,Scale=MatchLowercase}
%\setmainfont[Scale=.95]{Arial}
%\setmonofont{DejaVu Sans}

\usepackage{comment}

% include the `tex` instructions that takes care of loading packages and defining commands
% Copyright (c) 2012 Cies Breijs
%
% The MIT License
%
% Permission is hereby granted, free of charge, to any person obtaining a copy
% of this software and associated documentation files (the "Software"), to deal
% in the Software without restriction, including without limitation the rights
% to use, copy, modify, merge, publish, distribute, sublicense, and/or sell
% copies of the Software, and to permit persons to whom the Software is
% furnished to do so, subject to the following conditions:
%
% The above copyright notice and this permission notice shall be included in
% all copies or substantial portions of the Software.
%
% THE SOFTWARE IS PROVIDED "AS IS", WITHOUT WARRANTY OF ANY KIND, EXPRESS OR
% IMPLIED, INCLUDING BUT NOT LIMITED TO THE WARRANTIES OF MERCHANTABILITY,
% FITNESS FOR A PARTICULAR PURPOSE AND NONINFRINGEMENT. IN NO EVENT SHALL THE
% AUTHORS OR COPYRIGHT HOLDERS BE LIABLE FOR ANY CLAIM, DAMAGES OR OTHER
% LIABILITY, WHETHER IN AN ACTION OF CONTRACT, TORT OR OTHERWISE, ARISING FROM,
% OUT OF OR IN CONNECTION WITH THE SOFTWARE OR THE USE OR OTHER DEALINGS IN THE
% SOFTWARE.


% Some commands for making a LaTeX resume
% =======================================

% Commented ;)

% See the README.md file for more info



% \documentclass[10pt,a4paper]{article}  % i do this in the document itself


%%% LOAD AND SETUP PACKAGES

\usepackage[a4paper,margin=0.6in]{geometry}
%\usepackage{mdwlist}   % to finetue lists with a inline heading and indented content (see Experiences)
%\usepackage{showframe}
\usepackage{multicol}  % for multiple column text
\usepackage{parcolumns}  % for multiple column text
\usepackage{relsize}   % for \textscale, which I prefer over \sc (small caps), see my \acr command
%\usepackage[english]{babel}
%\hyphenation{Some-long-word}

%\usepackage[pdftex]{hyperref}  % yups, URLs everwhere...
\usepackage{hyperref}  % yups, URLs everwhere...
\usepackage{xcolor}  % ... and color them links
\definecolor{dark-blue}{rgb}{0.15,0.15,0.4}
\hypersetup{colorlinks,linkcolor={dark-blue},citecolor={dark-blue},urlcolor={dark-blue}}

%\usepackage{ifxetex}
%\ifxetex
%  \usepackage{fontspec}
%  \setmainfont
%    [ ExternalLocation ,
%      Mapping          = tex-text ,
%      Numbers          = OldStyle ,
%      Ligatures        = {Common,Contextual} ,
%      BoldFont         = texgyrepagella-bold.otf ,
%      ItalicFont       = texgyrepagella-italic.otf ,
%      BoldItalicFont   = texgyrepagella-bolditalic.otf ]
%    {texgyrepagella-regular.otf}
%  % Comment out the previous statement and uncomment the following line to use the
%  % Linux Libertine font (it has nice lignatures).
%  % Make sure to have the `ttf-linux-libertine` package installed on Ubuntu.
%%  \setmainfont[Mapping=tex-text, Numbers=OldStyle, Ligatures={Common,Contextual}]{Linux Libertine O}
%  \usepackage[protrusion]{microtype}  % needs an experimental and impposible to find package for xetex
%\else
%  \usepackage{tgpagella}  % this case we lack lower case numbers, ligatures and some typographic niceties
%  \usepackage[expansion,protrusion]{microtype}
%\fi

%%% BZ PACKAGES AND STYLING
\usepackage{enumitem}   % list package (http://tex.stackexchange.com/questions/10684/vertical-space-in-lists)
\usepackage{xifthen}  %% The xifthen package provides the \ifthenelse construct and the \isempty test

%\setlist{nolistsep,topsep=0pt,parsep=0pt,partopsep=0pt}
\setlist{nolistsep}
\parindent=0pt


%%% DOCUMENT WIDE STYLING

\pagestyle{empty}
\setlength{\tabcolsep}{0em}
\xspaceskip7pt  % some more spacing between sentences (use "i.e.\ " or "with SQL\@. " in case of errors)


%%% CUSTOM COMMANDS

%%% MY OWN COMMANDS (bz)

\newcommand*\roottitlelined[1] {
    \spacedhrule{0.8em}{-0.5em}  % a horizontal line with some vertical spacing before and after
    \subsection*{#1}
    \vspace{-0.3em}
    \nopagebreak[4]
    \vspace{-0.5em}  % some vertical spacing
}

% fucking environment, I fucking HATE YOU!
\newenvironment{indented} {
    \begin{list} {} {
        %\parsep=0pt\topsep=0pt\partopsep=0pt
        %\setlength{\leftmargin}{\newparindent}
        \setlength{\parsep}{0pt}
        \setlength{\parskip}{0pt}
        \setlength{\itemsep}{0pt}
        \setlength{\topsep}{0pt}
        \setlength{\leftmargin}{15pt}
        \setlength{\rightmargin}{0pt}
    }
} {
    \end{list}
}

\newcommand{\job}[3]{
    \vspace{0.3em}
    \nopagebreak[4]
    \textscale{1.1}{#1}
    \hfill #2

    \vspace{-0.2em}
    \begin{indented}
        \item[]
        #3
    \end{indented}
    \nopagebreak[4]
    \vspace{-0.2em}
}

\newcommand{\position}[3] {
    \nopagebreak[4]
    \textbf{#1}             % position
    \hfill \emph{#2}        % time period

    % do not waste space for content is it wasn't given
    \ifthenelse{\isempty{#3}}
        {}
        {
            \vspace{-0.2em}
            \begin{indented}
                \item[]
                #3
            \end{indented}
        }
    \pagebreak[2]
}

% put position and contents into the same line
\newcommand{\positionnobreak}[3] {
    \nopagebreak[4]
    \textbf{#1}, #3         % position + contents
    \hfill \emph{#2}        % time period

    \pagebreak[2]
}

\newcommand{\course}[3] {
    \textbf{#1} & #2 & \emph{#3} \\
}


%%% COMMANDS OF THE ORIGINAL AUTHOR

% main title (name) with subtitle (date)
\newcommand*\maintitle[2]{\noindent{\LARGE \textbf{#1}}\ \ \ \emph{#2}}

% title for the root sections (experience, education, etc) of the resume
\newcommand*\roottitle[1]{\subsection*{#1}\vspace{-0.3em}\nopagebreak[4]}

% acr command, to quickly mark acronyms for special formatting
\newcommand*\acr[1]{\textscale{.85}{#1}}

% pretty bullet (created from a much smaller centerdot), \sbull contains its spacing
\newcommand*\bull{\raisebox{-0.365em}[-1em][-1em]{\textscale{4}{$\cdot$}}}
\newcommand*\sbull{\ \ \bull \ \ }

% it seems not to work when simply using \parindent...
\newlength{\newparindent}
%\addtolength{\newparindent}{\parindent}

% a double \parindent...
\newlength{\doubleparindent}
%\addtolength{\doubleparindent}{\parindent}
%\addtolength{\doubleparindent}{\parindent}

% indentsection style, used for sections that aren't already in lists
% that need indentation to the level of all text in the document
\newenvironment{indentsection} {
    \begin{list} {} {
        %\setlength{\leftmargin}{\newparindent}
        \setlength{\leftmargin}{0pt}
        \setlength{\parsep}{0pt}
        \setlength{\parskip}{0pt}
        \setlength{\itemsep}{0pt}
        \setlength{\topsep}{0pt}
    }
} {
    \end{list}
}

% same as indentsection but without leftmargin indent
% \newenvironment{flatsection} {
%     \begin{list} {} {
%         \setlength{\leftmargin}{0pt}
%         \setlength{\parsep}{0pt}
%         \setlength{\parskip}{0pt}
%         \setlength{\itemsep}{0pt}
%         \setlength{\topsep}{0pt}
%     }
% } {
%     \end{list}
% }

% headerrow command, used for a new employer
\newcommand{\headedsection}[3]{
    \nopagebreak[4]
    %\begin{indentsection}
    %    \item[]
        \textscale{1.1}{#1}
        \hfill
        #2
        #3
    %\end{indentsection}
    \nopagebreak[4]
}

% subheaderrow command, used for a new position
\newcommand{\headedsubsection}[3] {
    \nopagebreak[4]
    \begin{indentsection}
        \item[]
        \textbf{#1}
        \hfill
        \emph{#2}
        #3
    \end{indentsection}
    \nopagebreak[4]
}

% body text (indented)
\newcommand{\bodytext}[1]{
    \nopagebreak[4]
    \begin{indentsection}
        \item[]
        #1
    \end{indentsection}
    \pagebreak[2]
}

% \vspace variaties
\newcommand{\breakvspace}[1]{\pagebreak[2]\vspace{#1}\pagebreak[2]}
\newcommand{\nobreakvspace}[1]{\nopagebreak[4]\vspace{#1}\nopagebreak[4]}

% \spacedhrule a horizontal line with some vertical space before and after it
\newcommand{\spacedhrule}[2]{\breakvspace{#1}\hrule\nobreakvspace{#2}}
\newcommand{\spaceddotrule}[2]{\breakvspace{#1}\dotfill\nobreakvspace{#2}}

% \inlineheadsection command, used for a new employer
\newcommand{\inlineheadsection}[2] {
    \begin{enumerate}[align=left] {
        \setlength{\leftmargin}{\doubleparindent}
    }
    %\item[\hspace{\newparindent}\textbf{#1}]
    \item[\textbf{#1}]
    #2
    \end{enumerate}
    %\vspace{-1.5em}
}

% \inlineheadsection command, used for a new employer
%\newcommand{\inlineheadsection}[2] {
%    \begin{basedescript} {
%        \setlength{\leftmargin}{\doubleparindent}
%    }
%    \item[\hspace{\newparindent}\textbf{#1}]
%    #2
%    \end{basedescript}
%    \vspace{-1.5em}
%}

% apo command, for an apostrophe that looks good on old style nums
\newcommand{\apo}{\raisebox{-.18ex}{'}{\hspace{-.1em}}}

% non space that allows line breaks
\newcommand*{\nsp}{\hskip0pt}

%%% MORE SPECIFIC COMMANDS

% CPP command (found it in some corner of the internet and decided to use it)
\newcommand{\CPP}{C\nolinebreak[4]\hspace{-.04em}\raisebox{.20ex}{\footnotesize\bf++}}
\newcommand{\CS}{C\nolinebreak\hspace{-.04em}\raisebox{.30ex}{\footnotesize\bf\#}}

% KTurtle command :)
\newcommand{\KTurtle}{\acr{KT}urtle }



% % these are in the document itself:
%
% \begin{document}
% ...the document text...
% \end{document}


\begin{document}  % begin the content of the document
\sloppy  % this to relax whitespacing in favour of straight margins

% \maintitle{Yuri Bochkarev}{May 24, 1988}  % title on top of the document
\maintitle{Yuri Bochkarev}{}  % title on top of the document

\nobreakvspace{0.3em}  % add some page break averse vertical spacing

% \noindent prevents paragraph's first lines from indenting
% \mbox is used to obfuscate the email address
% \sbull is a spaced bullet
% \href well..
% \\ breaks the line into a new paragraph
Moscow (Russia)
\sbull
%\noindent
\href{mailto:baltazar.bz@gmail.com}{baltazar.bz\mbox{}@\mbox{}gmail.com}\sbull
\textsmaller{+}79853037426\sbull
%cies010 \emph{(Skype)}\sbull
\href{http://ru.linkedin.com/pub/yuri-bochkarev/21/3a9/555}{linkedin}
\sbull
\href{https://github.com/balta2ar}{github}
% \\
%Mathenesserplein 84\sbull
%3022\thinspace {\sc ld}\sbull
%\spacedhrule{0.8em}{-0.4em}  % a horizontal line with some vertical spacing before and after

\roottitlelined{Summary}  % a root section title

%\vspace{0em}  % some vertical spacing

%\begin{multicols}{2}  % open a multicolumn environment
\noindent
Linux Python backend developer with 10+ years of experience in
    building, testing and maintaining complex software systems. 6+ years of
    Python experience, 6+ years of C++ experience, low-level background.
    Driven to get things done as well as to learn new skills, tools and
    technologies. Skilled to write efficient, maintainable and testable code.

    % Able
    % to communicate effectively and clearly in Russian and English both spoken
    % and written.

%\end{multicols}
%\spacedhrule{0.6em}{-0.4em}

\roottitlelined{Experience}

\job  % sets the header for the section and includes any subsections
    {\href{http://www.iponweb.com}{IPONWEB} -- media trading platform
     (\acr{RTB}, Ad exchange, \acr{DSP}, \acr{SSP})}
    {Moscow}
    {
        \position  % sets the header for a subsection and contains usually body text
        {Senior Software Developer (Internal Tools, \href{http://www.bidswitch.com/}{Bidswitch})}
        {June 2017 -- present}
        {
            \begin{itemize}
                \item{Transitioned the user management and access control subsystem
                      across departments that allowed for faster and simpler
                      workflow and operations (Python3.6, Django 1.11,
                      Django Rest Framework 3.6, Gabbi HTTP testing suite)}
                \item{Integrated Third-Party APIs with the internal
                      discrepancy monitoring service (Python3.6, PostgreSQL 9.6)}
                \item{Added support for bulk operations to internal creative
                      blocking service, which saved about 1 hour a day for client
                      support team (Python3.6, PostgreSQL 9.6)}
                \item{Integrated Third-Party API with the internal creative
                      approval service (Python3.5, asyncio, Cassandra 2.6)}
                \item{Optimized user group update microservice to reduce
                      processing time by 3x (Python 3.4, pytest, pymongo, TokuMX)}
                \item{Developed a facade API service to the internal DB as a part of
                      an integration with a client (Golang 1.8, Cassandra 2.6)}
                \item{Added metrics to the ``logs to DB'' loading tool, which provided
                      better understanding and visibility of the tool's internal state
                      (C++ 11, Cassandra 2.6)}
                \item{Software development process: Scrum, Kanban}
                \item{Technology stack: Docker, Kubenetes, Sentry, Jenkins,
                      Artifactory, OBS, Graphite, Grafana, Kibana, Ubuntu, Alpine}
            \end{itemize}
        }
        \position  % sets the header for a subsection and contains usually body text
            {Senior Software Developer (Technology)}
            {May 2016 -- June 2017}
            {
                \begin{itemize}
                    \item{Extended the functionality of the campaign visualization and
                          forecasting service (Python, Django, Numpy, Scipy, Pandas)}
                    \item{Integrated forecasting service functionality with the
                          custom testing stands}
                    \item{Software development process: Scrum + Kanban}
                \end{itemize}
            }
        \position  % sets the header for a subsection and contains usually body text
            {Software Developer (Technology)}
            {July 2014 -- May 2016}
            {
                \begin{itemize}
                    \item{Optimized network communication subsystem of the
                          budget control server for advertising campaigns, which
                          reduced total iteration time by 30\% (Python, Twisted, Numpy, Pandas)}
                    \item{Rolled out Continuous Integration infrastructure
                          (Jenkins, Trial, pytest, custom testing stands)}
                \end{itemize}
            }
        \position  % sets the header for a subsection and contains usually body text
            {Software Developer (R\&D)}
            {January 2013 -- July 2014}
            {
                \begin{itemize}
                    \item{Refactored and developed the data delivery service,
                          made it more modular, testable and robust
                          (Python, \acr{BASH}, Django)}
                    \item{Added performance/time metrics to comply with the established
                          SLA}
                    \item{Standardized data transfer configuration DSL that
                          is used across the company services}
                    \item{Optimized file transfer time by integrating UDP-based
                          file transfer solutions (UDT)}
                \end{itemize}
            }
    }


\job  % sets the header for the section and includes any subsections
    {\href{http://www.smartlabs.tv}{SmartLabs} -- interactive digital
           television (\acr{IPTV}, \acr{DVB}, \acr{OTT TV})}
    {Moscow}
    {
        \position % sets the header for a subsection and contains usually body text
            {System Department Developer}
            {September 2009 -- January 2013}
            {
                \begin{itemize}
                    \item{Developed an RTSP request redirector (\CPP, Qt)}
                    \item{Developed a video content distribution and publishing
                          automation tool (\CPP, Qt)}
                    \item{Integrated a third-party encryption service with the
                          video content distribution automation tool (VeriMatrix)}
                    \item{Improved SmartMedia platform video server seek time by
                          implementing a faster file segment lookup algorithm
                          (\CPP, Qt, \acr{STL}, live555)}
                    \item{Implemented an adaptive streaming client that was used
                          in Set-Top Boxes (\CPP, Qt, Apple HTTP Streaming)}
                    \item{Developed an interactive build automation tool that
                          helped developers save time and do fewer mistakes when
                          building platform components (Python)}
                    \item{Took part in the implementation of the adaptive
                          streaming server (Python, Twisted)}
                    \item{Accompanying protocols: \acr{HTTP}, \acr{XML RPC},
                          \acr{SOAP}, \acr{RTSP}, \acr{HLS}}
                \end{itemize}
            }
    }

%\vspace{0.7em}

\job
    {Autoscan}
    {\textsc{Ulyanovsk}}
    {
        \position
            {Software Developer}
            {September 2007 -- May 2009}
            {
                \begin{itemize}
                    \item{Prototyped and developed network applications for
                          real-time video transmission
                          (\CPP, Qt, \acr{OPAL}, libjingle)}
                    \item{Application testing automation (AutoIt, Erlang)}
                    \item{Accompanying protocols: \acr{XMPP}, \acr{SIP}, \acr{RTP},
                          \acr{STUN}}
                \end{itemize}
            }
    }

%\vspace{0.7em}

\job
    {\href{http://www.ulstu.ru}{Ulyanovsk State Technical University}}
    {\textsc{Ulyanovsk}}
    {
        \position
            {Technician}
            {September 2006 -- September 2007}
            {
                \begin{itemize}
                    \item{Computer class administration (11 computers,
                          \acr{ASP} Linux 9.0, Microsoft Windows 2003})
                    \item{Class video surveillance organization, technical support}
                \end{itemize}
            }
    }

%\spacedhrule{0.5em}{-0.4em}

\newpage

\roottitlelined{Education}

\job
    {\href{http://www.ulstu.ru}{Ulyanovsk State Technical University}}
    {\textsc{Ulyanovsk}}
    {
        \positionnobreak
            {Master's degree}
            {2007 -- 2009}
            {Faculty of Information Systems \& Technologies, diploma \acr{VMA} 0105774}
        \positionnobreak
            {Bachelor's degree}
            {2003 -- 2007}
            {Faculty of Information Systems \& Technologies, diploma \acr{AVB} 0512119}
    }

%\spacedhrule{0.5em}{-0.4em}
\roottitlelined{Additional Education}

\vspace{-0.7em}
\begin{minipage}[t]{0.48\textwidth}
\setlength\abovedisplayskip{0pt}
\courseplatform
    {\href{http://ita.by}{Information Technology Alliance}}
    {}
    {
        \course
            {\href{https://docs.google.com/file/d/0B2cptPgckn74RlpNa0xVNG9VSnM/edit}
                  {Clojure Course}}
            {April 2013}
            {}
    }
\courseplatform
    {\href{http://udacity.com}{udacity.com}}
    {}
    {
        \course
            {\href{https://docs.google.com/document/d/1LpUyUwh_gGyPyKf-oxTDOy8ncQejwog1jhgMmtf59mY/edit}
                  {Programming a Robotic Car}}
            {April 2012}
            {}
        \course
            {\href{https://docs.google.com/document/d/1wD_QEJ7mdzxbR_PMVEbZ_tZ0SyakJ_8Y1gBAj_S5Ufg/edit}
                  {Artificial Intelligence}}
            {December 2011}
            {}
    }
\courseplatform
    {\href{http://www.coursera.org}{coursera.org}}
    {}
    {
        \course
            {\href{https://docs.google.com/document/d/1Rh_5PSFQBBtEwMZIa5r69H5ONOjpX5jfiPorFIvQg5Y/edit}
                  {Computing for Data Analysis}}
            {October 2012}
            {}
        \course
            {\href{https://docs.google.com/document/d/1j6LlyJUGM03TxqSImyHeHoa16dSCLZQop6zmrPe8YOw/edit}
                  {Algorithms: Design and Analysis, Part I}}
            {April 2012}
            {}
        \course
            {\href{https://docs.google.com/document/d/11OT8thqIgBiwM80D_HjpiGtKTz5CnxiITPG_H6QbuUA/edit}
                  {Machine Learning}}
            {December 2011}
            {}
    }
\end{minipage}
\hfill
\begin{minipage}[t]{0.50\textwidth}
\setlength\abovedisplayskip{0pt}
\courseplatform
    {\href{http://www.coursera.org}{coursera.org}}
    {}
    {
        \course
            {\href{}
                  {Machine Learning: Classification}}
            {April 2016}
            {}
        \course
            {\href{}
                  {Game Theory (HSE)}}
            {March 2016}
            {}
        \course
            {\href{}
                  {Machine Learning: Regression}}
            {February 2016}
            {}
        \course
            {\href{}
                  {ML Foundations: A Case Study Approach}}
            {February 2016}
            {}
        \course
            {\href{}
                  {Cloud Computing Applications}}
            {October 2015}
            {}
        \course
            {\href{https://drive.google.com/file/d/0B2cptPgckn74MGQ5SUVQRGVDNE0/view}
                  {Image and Video Processing}}
            {March 2015}
            {}
        \course
            {\href{https://drive.google.com/file/d/0B2cptPgckn74ZkIwTWNFYUx6STA/view?usp=sharing}
                  {Algorithmic Thinking}}
            {November 2014}
            {}
        \course
            {\href{https://drive.google.com/file/d/0B2cptPgckn74OWlrVEZlR3U3azQ/edit?usp=sharing}
                  {Cryptography, Part I}}
            {June 2014}
            {}
        \course
            {\href{https://drive.google.com/file/d/0B2cptPgckn74WHJkQkNTdnN4dVU/edit?usp=sharing}
                  {Write Like Mozart}}
            {March 2014}
            {}
        \course
            {\href{https://docs.google.com/file/d/0B2cptPgckn74ZnN0Znhnd3ZUejg/edit?usp=sharing}
                  {Discrete Optimization}}
            {August 2013}
            {}
        \course
            {\href{https://docs.google.com/document/d/1AWaukQ0K4C-ZcuRRqYANJuYDOPxYH7MoVdXYKHPYJ8I/edit}
                  {Functional Programming Principles in Scala}}
            {December 2012}
            {}
    }
\end{minipage}

%\spacedhrule{0.5em}{-0.4em}

\roottitlelined{Skills}

\inlineheadsection  % special section that has an inline header with a 'hanging' paragraph
    {Technical}
    {
        \acr{OS}: \acr{GNU}/Linux (since 2009), Windows (2001 -- 2009).
        Programming Languages: Python, C++.
        Moderately experienced as well in: Go, Haskell, Clojure, R, Scala,
        Erlang, Lua, AutoIt, Java, Octave, x86 assembler.
        VCS: hg, git, svn.
        Technologies: Unit Testing, Design Patterns, \acr{UML}.
        Frameworks: Twisted, Pyramid, Django, Pandas, Scipy.
    }

\inlineheadsection
    {Language}
    {
        Russian \emph{(native)}, English \emph{(upper intermediate).}
    }

%\spacedhrule{1.6em}{-0.4em}
%
%\roottitle{Interests}
%
%\inlineheadsection
%  {}
%  {Online study courses, Bach fugues and bodybuilding.}

\end{document}
