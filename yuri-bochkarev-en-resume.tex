% LaTeX source of my resume
% =========================

% Commented for easy reuse... ;)

% See the `README.md` file for more info.

% This file is licensed under the CC-NC-ND Creative Commons license.


% start a document with the here given default font size and paper size
%\documentclass[10pt,a4paper]{article}
%\usepackage[T2A]{fontenc}
%\usepackage[russian,english]{babel}
%\usepackage[utf8x]{inputenc}

%\usepackage[cm-default]{fontspec}
%\usepackage{xltxtra}

%\usepackage[xetex]{graphicx}
%\usepackage{fontspec,xunicode}

%\defaultfontfeatures{Mapping=tex-text,Scale=MatchLowercase}

\documentclass[unicode, 10pt, a4paper, oneside, fleqn]{article}

%\usepackage[T1]{fontenc}

\usepackage{polyglossia}  %% подключает пакет многоязыкой вёрстки
\setmainfont[Ligatures=TeX]{Liberation Serif}  %% задаёт основной шрифт документа
%\setmainfont[Ligatures=TeX]{Linux Libertine O}  %% задаёт основной шрифт документа
%\setmainfont{PT Serif}  %% задаёт основной шрифт документа
%\setsansfont{Liberation Sans}  %% задаёт шрифт без засечек
%\setmonofont{Liberation Sans Mono}  %% задаёт моноширинный шрифт

%\setmainfont[Mapping=tex-text, Numbers=OldStyle, Ligatures={Common,Contextual}]{Linux Libertine O}

\defaultfontfeatures{Scale=MatchLowercase, Mapping=tex-text}  %% устанавливает поведение шрифтов по умолчанию
\setdefaultlanguage[spelling=modern]{russian}  %% устанавливает язык по умолчанию
\setotherlanguage{english}

%\fontsize{4mm}{5mm}\selectfont

%\renewcommand{\normalsize}{\fontsize{12.5}{10}\selectfont}

%\defaultfontfeatures{Mapping=tex-text,Scale=MatchLowercase}
%\setmainfont[Scale=.95]{Arial}
%\setmonofont{DejaVu Sans}

\usepackage{comment}

% include the `tex` instructions that takes care of loading packages and defining commands
% Copyright (c) 2012 Cies Breijs
%
% The MIT License
%
% Permission is hereby granted, free of charge, to any person obtaining a copy
% of this software and associated documentation files (the "Software"), to deal
% in the Software without restriction, including without limitation the rights
% to use, copy, modify, merge, publish, distribute, sublicense, and/or sell
% copies of the Software, and to permit persons to whom the Software is
% furnished to do so, subject to the following conditions:
%
% The above copyright notice and this permission notice shall be included in
% all copies or substantial portions of the Software.
%
% THE SOFTWARE IS PROVIDED "AS IS", WITHOUT WARRANTY OF ANY KIND, EXPRESS OR
% IMPLIED, INCLUDING BUT NOT LIMITED TO THE WARRANTIES OF MERCHANTABILITY,
% FITNESS FOR A PARTICULAR PURPOSE AND NONINFRINGEMENT. IN NO EVENT SHALL THE
% AUTHORS OR COPYRIGHT HOLDERS BE LIABLE FOR ANY CLAIM, DAMAGES OR OTHER
% LIABILITY, WHETHER IN AN ACTION OF CONTRACT, TORT OR OTHERWISE, ARISING FROM,
% OUT OF OR IN CONNECTION WITH THE SOFTWARE OR THE USE OR OTHER DEALINGS IN THE
% SOFTWARE.


% Some commands for making a LaTeX resume
% =======================================

% Commented ;)

% See the README.md file for more info



% \documentclass[10pt,a4paper]{article}  % i do this in the document itself


%%% LOAD AND SETUP PACKAGES

\usepackage[a4paper,margin=0.6in]{geometry}
%\usepackage{mdwlist}   % to finetue lists with a inline heading and indented content (see Experiences)
%\usepackage{showframe}
\usepackage{multicol}  % for multiple column text
\usepackage{parcolumns}  % for multiple column text
\usepackage{relsize}   % for \textscale, which I prefer over \sc (small caps), see my \acr command
%\usepackage[english]{babel}
%\hyphenation{Some-long-word}

%\usepackage[pdftex]{hyperref}  % yups, URLs everwhere...
\usepackage{hyperref}  % yups, URLs everwhere...
\usepackage{xcolor}  % ... and color them links
\definecolor{dark-blue}{rgb}{0.15,0.15,0.4}
\hypersetup{colorlinks,linkcolor={dark-blue},citecolor={dark-blue},urlcolor={dark-blue}}

%\usepackage{ifxetex}
%\ifxetex
%  \usepackage{fontspec}
%  \setmainfont
%    [ ExternalLocation ,
%      Mapping          = tex-text ,
%      Numbers          = OldStyle ,
%      Ligatures        = {Common,Contextual} ,
%      BoldFont         = texgyrepagella-bold.otf ,
%      ItalicFont       = texgyrepagella-italic.otf ,
%      BoldItalicFont   = texgyrepagella-bolditalic.otf ]
%    {texgyrepagella-regular.otf}
%  % Comment out the previous statement and uncomment the following line to use the
%  % Linux Libertine font (it has nice lignatures).
%  % Make sure to have the `ttf-linux-libertine` package installed on Ubuntu.
%%  \setmainfont[Mapping=tex-text, Numbers=OldStyle, Ligatures={Common,Contextual}]{Linux Libertine O}
%  \usepackage[protrusion]{microtype}  % needs an experimental and impposible to find package for xetex
%\else
%  \usepackage{tgpagella}  % this case we lack lower case numbers, ligatures and some typographic niceties
%  \usepackage[expansion,protrusion]{microtype}
%\fi

%%% BZ PACKAGES AND STYLING
\usepackage{enumitem}   % list package (http://tex.stackexchange.com/questions/10684/vertical-space-in-lists)
\usepackage{xifthen}  %% The xifthen package provides the \ifthenelse construct and the \isempty test

%\setlist{nolistsep,topsep=0pt,parsep=0pt,partopsep=0pt}
\setlist{nolistsep}
\parindent=0pt


%%% DOCUMENT WIDE STYLING

\pagestyle{empty}
\setlength{\tabcolsep}{0em}
\xspaceskip7pt  % some more spacing between sentences (use "i.e.\ " or "with SQL\@. " in case of errors)


%%% CUSTOM COMMANDS

%%% MY OWN COMMANDS (bz)

\newcommand*\roottitlelined[1] {
    \spacedhrule{0.8em}{-0.5em}  % a horizontal line with some vertical spacing before and after
    \subsection*{#1}
    \vspace{-0.3em}
    \nopagebreak[4]
    \vspace{-0.5em}  % some vertical spacing
}

% fucking environment, I fucking HATE YOU!
\newenvironment{indented} {
    \begin{list} {} {
        %\parsep=0pt\topsep=0pt\partopsep=0pt
        %\setlength{\leftmargin}{\newparindent}
        \setlength{\parsep}{0pt}
        \setlength{\parskip}{0pt}
        \setlength{\itemsep}{0pt}
        \setlength{\topsep}{0pt}
        \setlength{\leftmargin}{15pt}
        \setlength{\rightmargin}{0pt}
    }
} {
    \end{list}
}

\newcommand{\job}[3]{
    \vspace{0.3em}
    \nopagebreak[4]
    \textscale{1.1}{#1}
    \hfill #2

    \vspace{-0.2em}
    \begin{indented}
        \item[]
        #3
    \end{indented}
    \nopagebreak[4]
    \vspace{-0.2em}
}

\newcommand{\position}[3] {
    \nopagebreak[4]
    \textbf{#1}             % position
    \hfill \emph{#2}        % time period

    % do not waste space for content is it wasn't given
    \ifthenelse{\isempty{#3}}
        {}
        {
            \vspace{-0.2em}
            \begin{indented}
                \item[]
                #3
            \end{indented}
        }
    \pagebreak[2]
}

% put position and contents into the same line
\newcommand{\positionnobreak}[3] {
    \nopagebreak[4]
    \textbf{#1}, #3         % position + contents
    \hfill \emph{#2}        % time period

    \pagebreak[2]
}

\newcommand{\course}[3] {
    \textbf{#1} & #2 & \emph{#3} \\
}


%%% COMMANDS OF THE ORIGINAL AUTHOR

% main title (name) with subtitle (date)
\newcommand*\maintitle[2]{\noindent{\LARGE \textbf{#1}}\ \ \ \emph{#2}}

% title for the root sections (experience, education, etc) of the resume
\newcommand*\roottitle[1]{\subsection*{#1}\vspace{-0.3em}\nopagebreak[4]}

% acr command, to quickly mark acronyms for special formatting
\newcommand*\acr[1]{\textscale{.85}{#1}}

% pretty bullet (created from a much smaller centerdot), \sbull contains its spacing
\newcommand*\bull{\raisebox{-0.365em}[-1em][-1em]{\textscale{4}{$\cdot$}}}
\newcommand*\sbull{\ \ \bull \ \ }

% it seems not to work when simply using \parindent...
\newlength{\newparindent}
%\addtolength{\newparindent}{\parindent}

% a double \parindent...
\newlength{\doubleparindent}
%\addtolength{\doubleparindent}{\parindent}
%\addtolength{\doubleparindent}{\parindent}

% indentsection style, used for sections that aren't already in lists
% that need indentation to the level of all text in the document
\newenvironment{indentsection} {
    \begin{list} {} {
        %\setlength{\leftmargin}{\newparindent}
        \setlength{\leftmargin}{0pt}
        \setlength{\parsep}{0pt}
        \setlength{\parskip}{0pt}
        \setlength{\itemsep}{0pt}
        \setlength{\topsep}{0pt}
    }
} {
    \end{list}
}

% same as indentsection but without leftmargin indent
% \newenvironment{flatsection} {
%     \begin{list} {} {
%         \setlength{\leftmargin}{0pt}
%         \setlength{\parsep}{0pt}
%         \setlength{\parskip}{0pt}
%         \setlength{\itemsep}{0pt}
%         \setlength{\topsep}{0pt}
%     }
% } {
%     \end{list}
% }

% headerrow command, used for a new employer
\newcommand{\headedsection}[3]{
    \nopagebreak[4]
    %\begin{indentsection}
    %    \item[]
        \textscale{1.1}{#1}
        \hfill
        #2
        #3
    %\end{indentsection}
    \nopagebreak[4]
}

% subheaderrow command, used for a new position
\newcommand{\headedsubsection}[3] {
    \nopagebreak[4]
    \begin{indentsection}
        \item[]
        \textbf{#1}
        \hfill
        \emph{#2}
        #3
    \end{indentsection}
    \nopagebreak[4]
}

% body text (indented)
\newcommand{\bodytext}[1]{
    \nopagebreak[4]
    \begin{indentsection}
        \item[]
        #1
    \end{indentsection}
    \pagebreak[2]
}

% \vspace variaties
\newcommand{\breakvspace}[1]{\pagebreak[2]\vspace{#1}\pagebreak[2]}
\newcommand{\nobreakvspace}[1]{\nopagebreak[4]\vspace{#1}\nopagebreak[4]}

% \spacedhrule a horizontal line with some vertical space before and after it
\newcommand{\spacedhrule}[2]{\breakvspace{#1}\hrule\nobreakvspace{#2}}
\newcommand{\spaceddotrule}[2]{\breakvspace{#1}\dotfill\nobreakvspace{#2}}

% \inlineheadsection command, used for a new employer
\newcommand{\inlineheadsection}[2] {
    \begin{enumerate}[align=left] {
        \setlength{\leftmargin}{\doubleparindent}
    }
    %\item[\hspace{\newparindent}\textbf{#1}]
    \item[\textbf{#1}]
    #2
    \end{enumerate}
    %\vspace{-1.5em}
}

% \inlineheadsection command, used for a new employer
%\newcommand{\inlineheadsection}[2] {
%    \begin{basedescript} {
%        \setlength{\leftmargin}{\doubleparindent}
%    }
%    \item[\hspace{\newparindent}\textbf{#1}]
%    #2
%    \end{basedescript}
%    \vspace{-1.5em}
%}

% apo command, for an apostrophe that looks good on old style nums
\newcommand{\apo}{\raisebox{-.18ex}{'}{\hspace{-.1em}}}

% non space that allows line breaks
\newcommand*{\nsp}{\hskip0pt}

%%% MORE SPECIFIC COMMANDS

% CPP command (found it in some corner of the internet and decided to use it)
\newcommand{\CPP}{C\nolinebreak[4]\hspace{-.04em}\raisebox{.20ex}{\footnotesize\bf++}}
\newcommand{\CS}{C\nolinebreak\hspace{-.04em}\raisebox{.30ex}{\footnotesize\bf\#}}

% KTurtle command :)
\newcommand{\KTurtle}{\acr{KT}urtle }



% % these are in the document itself:
%
% \begin{document}
% ...the document text...
% \end{document}


\begin{document}  % begin the content of the document
\sloppy  % this to relax whitespacing in favour of straight margins

\maintitle{Yuri Bochkarev}{May 24, 1988}  % title on top of the document

\nobreakvspace{0.3em}  % add some page break averse vertical spacing

% \noindent prevents paragraph's first lines from indenting
% \mbox is used to obfuscate the email address
% \sbull is a spaced bullet
% \href well..
% \\ breaks the line into a new paragraph
\noindent\href{mailto:baltazar.dot.bz.at.gmail.dot.com}{baltazar.bz\mbox{}@\mbox{}gmail.com}\sbull
\textsmaller{+}79853037426\sbull
%cies010 \emph{(Skype)}\sbull
\href{http://ru.linkedin.com/pub/yuri-bochkarev/21/3a9/555}{linkedin}
\sbull
\href{https://bitbucket.org/balta2ar}{bitbucket}
\sbull
\href{https://github.org/balta2ar}{github}
\sbull
\href{http://baltazar-bz.blogspot.com/}{blog}
\\
%Mathenesserplein 84\sbull
%3022\thinspace {\sc ld}\sbull
Moscow\sbull
Russia
\spacedhrule{0.9em}{-0.4em}  % a horizontal line with some vertical spacing before and after

\roottitle{Summary}  % a root section title

\vspace{-1.3em}  % some vertical spacing

\begin{multicols}{2}  % open a multicolumn environment
\noindent 
1994 -- 2003: finished school (external studies).\\
2003 -- 2009: got bachelor's \& master's degrees in Ulyanovsk Technical State
University. \\
I work in Moscow since 2009.
\end{multicols}
\spacedhrule{0em}{-0.4em}

\roottitle{Experience}

\headedsection  % sets the header for the section and includes any subsections
  {\href{http://www.smartlabs.tv}{SmartLabs \acr{LLC}}}
  {\textsc{Moscow}} {%

  \headedsubsection  % sets the header for a subsection and contains usually body text
    {Developer in System Department}
    {September 2009 -- present}
    {\bodytext{
        \begin{itemize}
            \item{Development of user requests for video content viewing redirector (\CPP/Qt)}
            \item{Development of tool destined for video content distribution to client servers automation (\CPP/Qt)}
            \item{Integration of automation tool with existing platform}
            \item{Participation in Smart Media platform video server development
                  (\CPP/Qt, STL)}
              \item{Implementation of adaptive streaming (client side) (Apple HTTP Streaming)}
            \item{Development of tool for platform components assembly automation (python)}
            \item{Accompanying protocols: \acr{HTTP}, \acr{XML RPC},
                  \acr{SOAP}, \acr{RTSP}}
        \end{itemize}
    }}
}

\headedsection
{ \guillemotleft Autoscan\guillemotright {LLC}} 
  {\textsc{Ulyanovsk}} {
  \headedsubsection
    {Developer}
    {September 2007 -- May 2009}
    {\bodytext{
        \begin{itemize}
            \item{Development of network applications for real-time video transmission (\CPP/Qt)}
            \item{Automation of application's testing (AutoIt, erlang)}
            \item{Accompanying protocols: \acr{XMPP}, \acr{SIP}, \acr{RTP}, \acr{STUN}}
        \end{itemize}
    }}
}

\headedsection
  {\href{http://www.ulstu.ru}{Ulyanovsk Technical State University}}
  {\textsc{Ulyanovsk}} {
  \headedsubsection
    {Technician}
    {September 2006 -- September 2007}
    {\bodytext{
        \begin{itemize}
            \item{Administration of class (11 computers with servers based on \acr{ASP} Linux 9.0 и Microsoft Windows 2003})
            \item{Technical Support \& user consultation}
        \end{itemize}
    }}
}

\spacedhrule{0.5em}{-0.4em}
\roottitle{Education}

\headedsection
  {Ulyanovsk Technical State University}
  {\textsc{Ulyanovsk}} {
  \headedsubsection
    {Master degree}
    {2007 -- 2009} {
        \bodytext{Faculty of Information Systems \& Technologies, diploma \acr{VMA} 0105774}
    }
  }

\headedsection
  {Ulyanovsk Technical State University}
  {\textsc{Ulyanovsk}} {
  \headedsubsection
    {Bachelor degree}
    {2003 -- 2007} {
        \bodytext{Faculty of Information Systems \& Technologies, diploma \acr{AVB} 0512119}
    }
  }

\spacedhrule{0.5em}{-0.4em}
\roottitle{Additional Education}

\headedsection
  {\href{http://udacity.com}{udacity.com}}
  {} {
  \headedsubsection
    {\href{https://docs.google.com/document/d/1LpUyUwh_gGyPyKf-oxTDOy8ncQejwog1jhgMmtf59mY/edit}
          {Programming a Robotic Car}}
    {April, 2012} {}
  \headedsubsection
    {\href{https://docs.google.com/document/d/1wD_QEJ7mdzxbR_PMVEbZ_tZ0SyakJ_8Y1gBAj_S5Ufg/edit}
          {Algorithms: Design and Analysis, Part I}}
    {April, 2012} {}
  }
\headedsection
  {\href{http://www.coursera.org}{coursera.org}}
  {} {
  \headedsubsection
    {\href{https://docs.google.com/document/d/11OT8thqIgBiwM80D_HjpiGtKTz5CnxiITPG_H6QbuUA/edit}
          {Machine Learning}}
    {December, 2011} {}
  \headedsubsection
    {\href{https://docs.google.com/document/d/1wD_QEJ7mdzxbR_PMVEbZ_tZ0SyakJ_8Y1gBAj_S5Ufg/edit}
          {Articifial Intelligence}}
    {December, 2011} {}
  }

\spacedhrule{0.5em}{-0.4em}

\roottitle{Skills}

\inlineheadsection  % special section that has an inline header with a 'hanging' paragraph
  {Technical}
  {OS: Windows (95/98/ME/2000/XP/2003, 2001 -- 2009), Linux (ArchLinux, since 2009).
   Programming Languages: x86 assembler, C, C++, Python, Pascal, Object Pascal.
   Moderately experienced as well in: Haskell, Erlang, Clojure, Scala, Lua, AutoIt, \CS, Java,
   JavaScript. 
   Technologies: Test-Driven Development, Unit Testing, Design Patterns, UML.
  }
%  {Software design and implementation, with(in) a team. I love Ruby/\nsp
  %  Python/\nsp Java/\nsp \CPP and flirt regularly with Haskell. Solid
  %  knowledge of web technologies:\ \acr{HTML+CSS}, \acr{XML}, \acr{RDF},
  %  \acr{REST}, \acr{SOAP} and JavaScript (mainly jQuery). Linux administration
  %  skills:\ bash, Apache, My\acr{SQL}, Postgres\acr{SQL}, virtualization/cloud
  %  (Open\acr{VZ}, \acr{VM}ware, \acr{KVM}, Xen and \acr{EC}2), datacenter
  %  automation (Puppet and Chef), continuous integration (Hudson/Jenkins).}

\inlineheadsection
  {Language}
  {Russian \emph{(mother tongue)}, English \emph{(intermediate; fluent reading of technical documentation, technical writing).}}

\spacedhrule{1.6em}{-0.4em}

\roottitle{Interests}

\inlineheadsection
  {Bach fugues, bodybuilding, Haskell.}


\end{document}
