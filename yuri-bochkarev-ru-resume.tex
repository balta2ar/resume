% LaTeX source of my resume
% =========================
% originally from: https://github.com/cies/resume

% Commented for easy reuse... ;)

% See the `README.md` file for more info.

% This file is licensed under the CC-NC-ND Creative Commons license.


% start a document with the here given default font size and paper size
%\documentclass[10pt,a4paper]{article}
%\usepackage[T2A]{fontenc}
%\usepackage[russian,english]{babel}
%\usepackage[utf8x]{inputenc}

%\usepackage[cm-default]{fontspec}
%\usepackage{xltxtra}

%\usepackage[xetex]{graphicx}
%\usepackage{fontspec,xunicode}

%\defaultfontfeatures{Mapping=tex-text,Scale=MatchLowercase}

\documentclass[unicode, 10pt, a4paper, oneside, fleqn]{article}

%\usepackage[T1]{fontenc}

\usepackage{polyglossia}  %% подключает пакет многоязыкой вёрстки
\setmainfont[Ligatures=TeX]{Liberation Serif}  %% задаёт основной шрифт документа
%\setmainfont[Ligatures=TeX]{Linux Libertine O}  %% задаёт основной шрифт документа
%\setmainfont{PT Serif}  %% задаёт основной шрифт документа
%\setsansfont{Liberation Sans}  %% задаёт шрифт без засечек
%\setmonofont{Liberation Sans Mono}  %% задаёт моноширинный шрифт

%\setmainfont[Mapping=tex-text, Numbers=OldStyle, Ligatures={Common,Contextual}]{Linux Libertine O}

\defaultfontfeatures{Scale=MatchLowercase, Mapping=tex-text}  %% устанавливает поведение шрифтов по умолчанию
\setdefaultlanguage[spelling=modern]{russian}  %% устанавливает язык по умолчанию
\setotherlanguage{english}

%\fontsize{4mm}{5mm}\selectfont

%\renewcommand{\normalsize}{\fontsize{12.5}{10}\selectfont}

%\defaultfontfeatures{Mapping=tex-text,Scale=MatchLowercase}
%\setmainfont[Scale=.95]{Arial}
%\setmonofont{DejaVu Sans}

\usepackage{comment}

% include the `tex` instructions that takes care of loading packages and defining commands
\include{resume-commands}

\begin{document}  % begin the content of the document
\sloppy  % this to relax whitespacing in favour of straight margins

\maintitle{Юрий Бочкарев}{24 мая, 1988}  % title on top of the document

\nobreakvspace{0.3em}  % add some page break averse vertical spacing

% \noindent prevents paragraph's first lines from indenting
% \mbox is used to obfuscate the email address
% \sbull is a spaced bullet
% \href well..
% \\ breaks the line into a new paragraph
\noindent\href{mailto:baltazar.dot.bz.at.gmail.dot.com}{baltazar.bz\mbox{}@\mbox{}gmail.com}\sbull
\textsmaller{+}79853037426\sbull
%cies010 \emph{(Skype)}\sbull
\href{http://ru.linkedin.com/pub/yuri-bochkarev/21/3a9/555}{linkedin}
\sbull
\href{https://bitbucket.org/balta2ar}{bitbucket}
\sbull
\href{https://github.com/balta2ar}{github}
\sbull
\href{http://balta2ar.github.com}{github pages}
\sbull
\href{http://baltazar-bz.blogspot.com/}{blogspot}
\\
%Mathenesserplein 84\sbull
%3022\thinspace {\sc ld}\sbull
Москва\sbull
Россия
\spacedhrule{0.8em}{-0.4em}  % a horizontal line with some vertical spacing before and after

\roottitle{О себе}  % a root section title

\vspace{0em}  % some vertical spacing

%\begin{multicols}{1}  % open a multicolumn environment
\noindent 
1994 -- 2003: экстерном закончил школу. 2003 -- 2009: закончил бакалавриат и
магистратуру в Ульяновском Государственном Техническом Университете. Работаю в
Москве с 2009 года.
%\end{multicols}
\spacedhrule{0.6em}{-0.4em}

\roottitle{Опыт работы}

\headedsection  % sets the header for the section and includes any subsections
  {\href{http://www.smartlabs.tv}{SmartLabs \acr{LLC}} -- цифровое
   интерактивное телевидение (IPTV, DVB, OTT TV)}
  {Москва} {
  \headedsubsection  % sets the header for a subsection and contains usually body text
    {Разработчик системного отдела}
    {сентябрь 2009 -- настоящее время}
    {\bodytext {
        \begin{itemize}
            \item{Разработка редиректора пользовательских запросов на
                  просмотр видеоконтента (\CPP, Qt)}
            \item{Разработка средства автоматизации распространения
                  видеоконтента на серверы клиента (\CPP, Qt)}
            \item{Интеграция средства автоматизации с существующей платформой}
            \item{Участие в разработке видеосервера платформы SmartMedia
                  (\CPP, Qt, STL, live555)}
            \item{Реализация клиента адаптивного стриминга (\CPP, Qt, Apple HTTP Streaming)}
            \item{Разработка средства автоматизации сборки компонентов платформы
                  (python)}
            \item{Участие в разработке сервера адаптивного стриминга (python)}
            \item{Сопутствующие протоколы: \acr{HLS}, \acr{HTTP}, \acr{XML RPC},
                  \acr{SOAP}, \acr{RTSP}}
        \end{itemize}
    }}
}

%\spaceddotrule{-0.5em}{0em}
\vspace{0.7em}

\headedsection
  {\acr{ООО} "Автоскан"}
  {Ульяновск} {
  \headedsubsection
    {Разработчик}
    {сентябрь 2007 -- май 2009}
    {\bodytext {
        \begin{itemize}
            \item{Разработка сетевого приложения передачи видео в реальном
                  времени (\CPP, Qt, OPAL, libjingle)}
            \item{Автоматизация тестирования приложения (AutoIt, erlang)}
            \item{Сопутствующие протоколы: \acr{XMPP}, \acr{SIP}, \acr{RTP},
                  \acr{STUN}}
        \end{itemize}
    }}
}

%\spaceddotrule{-0.5em}{0em}
\vspace{0.7em}

\headedsection
  {\href{http://www.ulstu.ru}{Ульяновский Государственный Технический Университет}}
  {Ульяновск} {
  \headedsubsection
    {Техник}
    {сентябрь 2006 -- сентябрь 2007}
    {\bodytext {
        \begin{itemize}
            \item{Администрирование класса из 11 компьютеров с серверами на
                  базе \acr{ASP} Linux 9.0 и Microsoft Windows 2003}
            \item{Техническая поддержка и консультация пользователей}
        \end{itemize}
    }}
}

\spacedhrule{0.5em}{-0.4em}
\roottitle{Образование}

\headedsection
  {Ульяновский Государственный Технический Университет}
  {Ульяновск} {
  \headedsubsection
    {степень магистра}
    {2007 -- 2009} {
        \bodytext{Факультет Информационных Систем и Технологий, диплом \acr{ВМА} 0105774}
    }
  }

\headedsection
  {Ульяновский Государственный Технический Университет}
  {Ульяновск} {
  \headedsubsection
    {степень бакалавра}
    {2003 -- 2007} {
        \bodytext{Факультет Информационных Систем и Технологий, диплом \acr{АВБ} 0512119}
    }
  }

\spacedhrule{0.5em}{-0.4em}
\roottitle{Дополнительное образование}

\headedsection
  {\href{http://www.coursera.org}{coursera.org}}
  {} {
  \headedsubsection
    {\href{https://docs.google.com/document/d/1wD_QEJ7mdzxbR_PMVEbZ_tZ0SyakJ_8Y1gBAj_S5Ufg/edit}
          {Algorithms: Design and Analysis, Part I}}
    {апрель, 2012} {}
  \headedsubsection
    {\href{https://docs.google.com/document/d/11OT8thqIgBiwM80D_HjpiGtKTz5CnxiITPG_H6QbuUA/edit}
          {Machine Learning}}
    {декабрь, 2011} {}
  }

\headedsection
  {\href{http://udacity.com}{udacity.com}}
  {} {
  \headedsubsection
    {\href{https://docs.google.com/document/d/1LpUyUwh_gGyPyKf-oxTDOy8ncQejwog1jhgMmtf59mY/edit}
          {Programming a Robotic Car}}
    {апрель, 2012} {}

  \headedsubsection
    {\href{https://docs.google.com/document/d/1wD_QEJ7mdzxbR_PMVEbZ_tZ0SyakJ_8Y1gBAj_S5Ufg/edit}
          {Articifial Intelligence}}
    {декабрь, 2011} {}
  }

\spacedhrule{0.5em}{-0.4em}

\roottitle{Навыки}

\inlineheadsection  % special section that has an inline header with a 'hanging' paragraph
  {Технические}
  {\acr{ОС}: \acr{GNU}/Linux (c 2009), Windows (2001 -- 2009).
   Основные языки программирования: C++, Python.
   Также небольшой опыт: Haskell, Erlang, Clojure, Scala, Lua, AutoIt,
   Java, Octave, R, x86 assembler.
   Технологии: Unit Testing, паттерны проектирования, \acr{UML}.
   Фреймворки: небольшой опыт работы с Twisted, Pyramid.
  }

\inlineheadsection
  {Языковые}
  {Русский \emph{(носитель языка)}, English \emph{(средний: свободное чтение
   технической документации, письменная техническая речь)}}

\spacedhrule{1.6em}{-0.4em}

\roottitle{Интересы}

\inlineheadsection
  {}
  {Свободное время уделяю онлайн-курсам, фугам Баха и штанге.}

\end{document}
