% LaTeX source of my resume
% =========================
% originally from: https://github.com/cies/resume

% Commented for easy reuse... ;)

% See the `README.md` file for more info.

% This file is licensed under the CC-NC-ND Creative Commons license.


% start a document with the here given default font size and paper size
%\documentclass[10pt,a4paper]{article}
%\usepackage[T2A]{fontenc}
%\usepackage[russian,english]{babel}
%\usepackage[utf8x]{inputenc}

%\usepackage[cm-default]{fontspec}
%\usepackage{xltxtra}

%\usepackage[xetex]{graphicx}
%\usepackage{fontspec,xunicode}

%\defaultfontfeatures{Mapping=tex-text,Scale=MatchLowercase}

\documentclass[unicode, 10pt, a4paper, oneside, fleqn]{article}

%\usepackage[T1]{fontenc}

\usepackage{polyglossia}  %% подключает пакет многоязыкой вёрстки
\setmainfont[Ligatures=TeX]{Liberation Serif}  %% задаёт основной шрифт документа
%\setmainfont[Ligatures=TeX]{Linux Libertine O}  %% задаёт основной шрифт документа
%\setmainfont{PT Serif}  %% задаёт основной шрифт документа
%\setsansfont{Liberation Sans}  %% задаёт шрифт без засечек
%\setmonofont{Liberation Sans Mono}  %% задаёт моноширинный шрифт

%\setmainfont[Mapping=tex-text, Numbers=OldStyle, Ligatures={Common,Contextual}]{Linux Libertine O}

\defaultfontfeatures{Scale=MatchLowercase, Mapping=tex-text}  %% устанавливает поведение шрифтов по умолчанию
\setdefaultlanguage[spelling=modern]{russian}  %% устанавливает язык по умолчанию
\setotherlanguage{english}

%\fontsize{4mm}{5mm}\selectfont

%\renewcommand{\normalsize}{\fontsize{12.5}{10}\selectfont}

%\defaultfontfeatures{Mapping=tex-text,Scale=MatchLowercase}
%\setmainfont[Scale=.95]{Arial}
%\setmonofont{DejaVu Sans}

\usepackage{comment}

% include the `tex` instructions that takes care of loading packages and defining commands
\include{resume-commands}

\begin{document}  % begin the content of the document
\sloppy  % this to relax whitespacing in favour of straight margins

\maintitle{Юрий Бочкарев}{24 мая, 1988}  % title on top of the document

\nobreakvspace{0.3em}  % add some page break averse vertical spacing

% \noindent prevents paragraph's first lines from indenting
% \mbox is used to obfuscate the email address
% \sbull is a spaced bullet
% \href well..
% \\ breaks the line into a new paragraph
\noindent\href{mailto:baltazar.bz@gmail.com}{baltazar.bz\mbox{}@\mbox{}gmail.com}\sbull
\textsmaller{+}79853037426\sbull
%cies010 \emph{(Skype)}\sbull
\href{http://ru.linkedin.com/pub/yuri-bochkarev/21/3a9/555}{linkedin}
\sbull
\href{https://bitbucket.org/balta2ar}{bitbucket}
\sbull
\href{https://github.com/balta2ar}{github}
\sbull
\href{http://balta2ar.github.com}{github pages}
\sbull
\href{http://baltazar-bz.blogspot.com/}{blogspot}
\\
%Mathenesserplein 84\sbull
%3022\thinspace {\sc ld}\sbull
Москва\sbull
Россия
%\spacedhrule{0.8em}{-0.4em}  % a horizontal line with some vertical spacing before and after

\roottitlelined{О себе}  % a root section title

%\vspace{0em}  % some vertical spacing
%\begin{multicols}{1}  % open a multicolumn environment
\noindent 
1994 -- 2003: экстерном закончил школу. 2003 -- 2009: закончил бакалавриат и
магистратуру в Ульяновском Государственном Техническом Университете. Работаю в
Москве с 2009 года.

%\end{multicols}
%\spacedhrule{0.6em}{-0.4em}

\roottitlelined{Опыт работы}

\job  % sets the header for the section and includes any subsections
    {\href{http://www.iponweb.com}{IPONWEB} -- платформа предоставления
     Интернет-рекламы (\acr{RTB}, Ad exchange, \acr{DSP}, \acr{SSP})}
    {Москва}
    {
        \position  % sets the header for a subsection and contains usually body text
            {Разработчик отдела R\&D}
            {январь 2013 -- настоящее время}
            {
                \begin{itemize}
                \item{Поддержка и рефакторинг архитектуры сервиса доставки
                      данных (Python, \acr{BASH}, Django)}
                \item{Реализация бекенда сервиса визуализации данных
                      (Python, Django)}
                \item{Участие в разработке сервера управления бюджетом в
                      рекламных кампаниях (Python, Twisted, Pandas)}
                \end{itemize}
            }
    }

%\spaceddotrule{-0.5em}{0em}
%\vspace{0.7em}

\job  % sets the header for the section and includes any subsections
    {\href{http://www.smartlabs.tv}{SmartLabs} -- цифровое
     интерактивное телевидение (\acr{IPTV}, \acr{DVB}, \acr{OTT TV})}
    {Москва}
    {
        \position  % sets the header for a subsection and contains usually body text
            {Разработчик системного отдела}
            {сентябрь 2009 -- январь 2013}
            {
                \begin{itemize}
                    \item{Разработка редиректора пользовательских запросов на
                          просмотр видеоконтента (\CPP, Qt)}
                    \item{Разработка средства автоматизации распространения
                          видеоконтента на серверы клиента (\CPP, Qt)}
                    \item{Интеграция средства автоматизации с существующей платформой}
                    \item{Участие в разработке видеосервера платформы SmartMedia
                          (\CPP/Qt, \acr{STL})}
                    \item{Реализация клиента адаптивного стриминга (\CPP, Qt, Apple HTTP Streaming)}
                    \item{Разработка средства автоматизации сборки компонентов платформы
                          (Python)}
                    \item{Участие в разработке сервера адаптивного стриминга (Python)}
                    \item{Сопутствующие протоколы: \acr{HLS}, \acr{HTTP}, \acr{XML RPC},
                          \acr{SOAP}, \acr{RTSP}}
                \end{itemize}
            }
    }

%\spaceddotrule{-0.5em}{0em}
%\vspace{0.7em}

\job
    {Автоскан}
    {Ульяновск}
    {
        \position
            {Разработчик}
            {сентябрь 2007 -- май 2009}
            {
                \begin{itemize}
                    \item{Разработка сетевого приложения передачи видео в реальном
                          времени (\CPP, Qt, \acr{OPAL}, libjingle)}
                    \item{Автоматизация тестирования приложения (AutoIt, Erlang)}
                    \item{Сопутствующие протоколы: \acr{XMPP}, \acr{SIP}, \acr{RTP},
                          \acr{STUN}}
                \end{itemize}
            }
    }


%\spaceddotrule{-0.5em}{0em}
%\vspace{0.7em}

\job
    {\href{http://www.ulstu.ru}{Ульяновский Государственный Технический Университет}}
    {Ульяновск}
    {
        \position
            {Техник}
            {сентябрь 2006 -- сентябрь 2007}
            {
                \begin{itemize}
                    \item{Администрирование класса (11 компьютеров,
                          \acr{ASP} Linux 9.0 и Microsoft Windows 2003)}
                    \item{Организация системы видеонаблюдения в классе}
                    \item{Техническая поддержка и консультация пользователей}
                \end{itemize}
            }
    }

%\spacedhrule{0.5em}{-0.4em}
\roottitlelined{Образование}

\job
    {\href{http://www.ulstu.ru}{Ульяновский Государственный Технический Университет}}
    {Ульяновск}
    {
        \positionnobreak
            {степень магистра}
            {2007 -- 2009}
            {Факультет Информационных Систем и Технологий, диплом \acr{ВМА} 0105774}
        \positionnobreak
            {степень бакалавра}
            {2003 -- 2007}
            {Факультет Информационных Систем и Технологий, диплом \acr{АВБ} 0512119}
    }

%\spacedhrule{0.5em}{-0.4em}
\roottitlelined{Дополнительное образование}

% \begin{tabular*}{\textwidth}{l @{\extracolsep{\fill}} lr}
% \course
%     {\href{https://docs.google.com/file/d/0B2cptPgckn74RlpNa0xVNG9VSnM/edit}
%           {Clojure Course}}
%     {\href{http://ita.by}{Information Technology Alliance}}
%     {апрель 2013}
% \course
%     {\href{https://docs.google.com/document/d/1LpUyUwh_gGyPyKf-oxTDOy8ncQejwog1jhgMmtf59mY/edit}
%           {Programming a Robotic Car}}
%     {\href{http://udacity.com}{udacity.com}}
%     {апрель 2012}
% \course
%     {\href{https://docs.google.com/document/d/1wD_QEJ7mdzxbR_PMVEbZ_tZ0SyakJ_8Y1gBAj_S5Ufg/edit}
%           {Artificial Intelligence}}
%     {}
%     {декабрь 2011}
% \course
%     {\href{https://docs.google.com/document/d/1AWaukQ0K4C-ZcuRRqYANJuYDOPxYH7MoVdXYKHPYJ8I/edit}
%           {Functional Programming Principles in Scala}}
%     {\href{http://www.coursera.org}{coursera.org}}
%     {декабрь 2012}
% \course
%     {\href{https://docs.google.com/document/d/1Rh_5PSFQBBtEwMZIa5r69H5ONOjpX5jfiPorFIvQg5Y/edit}
%           {Computing for Data Analysis}}
%     {}
%     {октябрь 2012}
% \course
%     {\href{https://docs.google.com/document/d/1j6LlyJUGM03TxqSImyHeHoa16dSCLZQop6zmrPe8YOw/edit}
%           {Algorithms: Design and Analysis, Part I}}
%     {}
%     {апрель 2012}
% \course
%     {\href{https://docs.google.com/document/d/11OT8thqIgBiwM80D_HjpiGtKTz5CnxiITPG_H6QbuUA/edit}
%           {Machine Learning}}
%     {}
%     {декабрь 2011}
% \end{tabular*}

\vspace{-0.7em}
\begin{minipage}[t]{0.43\textwidth}
\setlength\abovedisplayskip{0pt}
\courseplatform
    {\href{http://ita.by}{Information Technology Alliance}}
    {}
    {
        \course
            {\href{https://docs.google.com/file/d/0B2cptPgckn74RlpNa0xVNG9VSnM/edit}
                  {Clojure Course}}
            {апрель 2013}
            {}
    }
\courseplatform
    {\href{http://udacity.com}{udacity.com}}
    {}
    {
        \course
            {\href{https://docs.google.com/document/d/1LpUyUwh_gGyPyKf-oxTDOy8ncQejwog1jhgMmtf59mY/edit}
                  {Programming a Robotic Car}}
            {апрель 2012}
            {}
        \course
            {\href{https://docs.google.com/document/d/1wD_QEJ7mdzxbR_PMVEbZ_tZ0SyakJ_8Y1gBAj_S5Ufg/edit}
                  {Artificial Intelligence}}
            {декабрь 2011}
            {}
    }
\end{minipage}
\hfill
\begin{minipage}[t]{0.55\textwidth}
\setlength\abovedisplayskip{0pt}
\courseplatform
    {\href{http://www.coursera.org}{coursera.org}}
    {}
    {
        \course
            {\href{}
                  {ML Foundations: A Case Study Approach}}
            {февраль 2016}
            {}
        \course
            {\href{}
                  {Cloud Computing Applications}}
            {октябрь 2015}
            {}
        \course
            {\href{https://drive.google.com/file/d/0B2cptPgckn74MGQ5SUVQRGVDNE0/view}
                  {Image and Video Processing}}
            {март 2015}
            {}
        \course
            {\href{https://drive.google.com/file/d/0B2cptPgckn74ZkIwTWNFYUx6STA/view?usp=sharing}
                  {Algorithmic Thinking}}
            {ноябрь 2014}
            {}
        \course
            {\href{https://drive.google.com/file/d/0B2cptPgckn74OWlrVEZlR3U3azQ/edit?usp=sharing}
                  {Cryptography, Part I}}
            {июнь 2014}
            {}
        \course
            {\href{https://drive.google.com/file/d/0B2cptPgckn74WHJkQkNTdnN4dVU/edit?usp=sharing}
                  {Write Like Mozart}}
            {март 2014}
            {}
        \course
            {\href{https://docs.google.com/file/d/0B2cptPgckn74ZnN0Znhnd3ZUejg/edit?usp=sharing}
                  {Discrete Optimization}}
            {август 2013}
            {}
        \course
            {\href{https://docs.google.com/document/d/1AWaukQ0K4C-ZcuRRqYANJuYDOPxYH7MoVdXYKHPYJ8I/edit}
                  {Functional Programming Principles in Scala}}
            {декабрь 2012}
            {}
        \course
            {\href{https://docs.google.com/document/d/1Rh_5PSFQBBtEwMZIa5r69H5ONOjpX5jfiPorFIvQg5Y/edit}
                  {Computing for Data Analysis}}
            {октябрь 2012}
            {}
        \course
            {\href{https://docs.google.com/document/d/1j6LlyJUGM03TxqSImyHeHoa16dSCLZQop6zmrPe8YOw/edit}
                  {Algorithms: Design and Analysis, Part I}}
            {апрель 2012}
            {}
        \course
            {\href{https://docs.google.com/document/d/11OT8thqIgBiwM80D_HjpiGtKTz5CnxiITPG_H6QbuUA/edit}
                  {Machine Learning}}
            {декабрь 2011}
            {}
    }
\end{minipage}

%scala https://docs.google.com/document/d/18Z2Y-FqLYMQsPQPVDYRJ64X_qyuaYLOP3Te7ScyfA6M
%comp for data analysis https://docs.google.com/document/d/1boL0vFYLzqr3UuSusrcKH1w6QVtqybHBYj03neW4qO8

%\spacedhrule{0.5em}{-0.4em}

\roottitlelined{Навыки}

\inlineheadsection  % special section that has an inline header with a 'hanging' paragraph
    {Технические}
    {
        \acr{ОС}: \acr{GNU}/Linux (c 2009), Windows (2001 -- 2009).
        Основные языки программирования: Python, \acr{C++}.
        Также небольшой опыт: Go, Haskell, Clojure, R, Scala, Erlang, Lua,
        AutoIt, Java, Octave, x86 assembler.
        VCS: hg, git, svn.
        Технологии: Unit Testing, паттерны проектирования, \acr{UML}.
        Фреймворки: небольшой опыт работы с Twisted, Pyramid, Django.
    }

\inlineheadsection
    {Языковые}
    {
        Русский \emph{(носитель языка)}, English \emph{(свободное чтение
        технической документации, письменная техническая речь)}
    }

% --- OLD ----------------------------------------------------------------------

\end{document}
